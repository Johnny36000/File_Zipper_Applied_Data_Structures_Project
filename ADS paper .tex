\documentclass[conference]{IEEEtran}
\usepackage{graphicx}
\usepackage{lipsum} % for sample text, remove later

\begin{document}


\begin{titlepage}
    \centering
    \vspace*{3cm}
    {\LARGE \textbf{Solving Sudoku Puzzles Using Variants of Backtracking Algorithms}\par}
    \vspace{2cm}

    {\large
    \textbf{Authors:}\\[6pt]
    Michael Guirgis\\
    David Boules\\
    John Fahmy\\
    Jana Hassan\\
    Mariam Hassan\\[1.5cm]
    
    Department of Computer Science and Engineering\\
    The American University in Cairo, Egypt\\[1.5cm]
    
    \textbf{Course:} CSCE2211 — Applied Data Structures\\[6pt]
    \textbf{Instructor:} Dr. Ashraf AbdelRaouf\\[6pt]
    \textbf{Submission Date:} \\[2cm]
    }
    
    \vfill
    {\small This paper follows the IEEE format for academic and research writing.}
\end{titlepage}




\begin{abstract}
(tentative) In this paper, we discuss various algorithms used to solve different types of Sudoku puzzles, including the standard $9\times9$, mini Sudoku, and $16\times16$ variants. We begin by introducing the basic structure and rules of the game, followed by a comparison between heuristic and backtracking-based approaches. The study then focuses on detailed implementations of backtracking algorithms, specifically Depth-First Search (DFS) and Dancing Links (DLX). We conclude that Dancing Links provides the most efficient and flexible solution due to the use of a linked-list data structure, which allows for rapid constraint removal and restoration during the search process.
\end{abstract}

\end{document}